\documentclass[12pt,oneside]{book}
\usepackage{geometry}                		% See geometry.pdf to learn the layout options. There are lots.
\geometry{a4paper}                   			% ... or a4paper or a5paper or ... 
%\geometry{landscape}                		% Activate for for rotated page geometry
%\usepackage[parfill]{parskip}    		% Activate to begin paragraphs with an empty line rather than an indent
\usepackage{graphicx}				% Use pdf, png, jpg, or epsß with pdflatex; use eps in DVI mode
								% TeX will automatically convert eps --> pdf in pdflatex		
\usepackage{amssymb}

\usepackage[spanish]{babel}			% Permite que partes automáticas del documento aparezcan en castellano.
\usepackage[utf8]{inputenc}			% Permite escribir tildes y otros caracteres directamente en el .tex
\usepackage[T1]{fontenc}				% Asegura que el documento resultante use caracteres de una fuente apropiada.

\usepackage{hyperref}				% Permite poner urls y links dentro del documento

\title{SUDOKU \\ Manual de Usuario}
\author{José Velez \\ Diego Cabrera \\ Carlos Ramirez}
%\date{}							% Activate to display a given date or no date

\begin{document}
\maketitle
\tableofcontents


\chapter{Derechos del Autor}
La aplicaci\'on que simula el Juego de Sudoku fue creada por prop\'ositos acad\'emicos como proyecto del Primer Parcial de la materia de Lenguajes de Programación que dicta el Ing. Tibau en el presente semestre <I Termino-2013>.
\\La cual fue desarrollada en conjunto por \\-José Velez \\ -Diego Cabrera \\ -Carlos Ramirez

\chapter{Prefacio}

Este manual de usuario pretende brindar un conjunto de instrucciones al Usuario para facilitar el funcionamiento del juego SUDOKU. \ \\ \\ 
Por otro lado, también brinda guías para el juego y describe las diferentes instancias que tiene el mismo. \ \\ \\
Con el fin de aprovechar al máximo la información de este documento, para ejecutar la aplicación de SUDOKU el usuario debe tener en su poder un PC que tenga instalado un Sistema Operativo Windows (XP, Vista, 7) o Linux (preferible UBUNTU).


\chapter{Introducción}

El Sudoku  fue creado en 1979 por Howard Garns un arquitecto de primera y publicado en una revista de Rompecabezas Americano. El golpe moda en Japón fue en 1986, pero no tomaron el centro de la escena hasta el 2005, cuando sitios web, libros rompecabezas y la cobertura de los medios de comunicación, hicieron que  significativos juego de Sudoku se convirtieran en una sensación mundial.



\chapter{Guía del Juego}

En este capitulo explicaremos la reglas del Sudoku y se detallar\'a el uso del simulador del juego en las diferentes instancias que tiene el mismo. 


\section{Reglas}

El juego está compuesto por una cuadrícula de 9x9 casillas, partiendo de algunos números ya dispuestos en algunas de las casillas, hay que completar las casillas vacías con dígitos del 1 al 9 sin que se repitan por fila, columna o región. 

\ \\

Regla 1: Completar las casillas vacías con un solo número del 1 al 9. 

\ \\ 

Regla 2: En una misma fila no puede haber números repetidos. 

\ \\ 

Regla 3: En una misma columna no puede haber números repetidos.

\ \\ 

Regla 4: En una misma región no puede haber números repetidos. 

\ \\ 

Regla 5: la solución de un sudoku es única.

\begin{figure}[htbp]
\begin{center}
\includegraphics[width=.50\textwidth]{./imagenes/Reglas.png}
\caption{Reglas de Juego}
\label{Reglas}
\end{center}
\end{figure}


\section{Simulador}
\ \\ Al inciar la aplicacion se mostrará la siguiente pantalla:

\begin{figure}[htbp]
\begin{center}
\includegraphics[width=.70\textwidth]{./imagenes/Inicio1.png}
\caption{Inicio}
\label{Inicio}
\end{center}
\end{figure}

\ \\ En donde deberá digitar el nombre de usuario con el que desee registrar su partida, y escoger uno de los 3 niveles de dificultad. \\ <Nota: Si no se escoge ningún nivel, por defecto se registrará el nivel Juvenil.>

\begin{figure}[htbp]
\begin{center}
\includegraphics[width=.60\textwidth]{./imagenes/Inicio4.png}
\caption{Usuario y Nivel}
\label{Usuario y Nivel}
\end{center}
\end{figure}

\ \\ \ \\ 
Luego tenemos dos opciones: \\ 1. Presionar en \textbf{SCORE} para visualizar el ranking de las mejores puntuaciones, ó \\ 2. Presionar en \textbf{PLAY} para JUGAR. 

\begin{figure}[htbp]
\begin{center}
\includegraphics[width=.60\textwidth]{./imagenes/Controles.png}
\caption{Controles}
\label{Controles}
\end{center}
\end{figure} 

\ \\ \ \\
\textbf{Nota:} Para poder ver los \textbf{SCORE} no es necesario escribir un nombre de usuario, mientras que para poder \textbf{JUGAR} si es obligatorio escribir un nombre de usuario que además sea valido, caso contrario le apareceran los siguientes mensajes de error:

\begin{figure}[htbp]
\begin{center}
\includegraphics[width=.60\textwidth]{./imagenes/ErrorNoJugador.png}
\caption{Errores de Usuario}
\label{Errores de Usuario}
\end{center}
\end{figure} 

\ \\ \ \\ \ \\ \ \\
Si estamos en la pantalla inicial (Figura 4.2), y si presionamos en \textbf{SCORE}, se mostrará la siguiente pantalla, en donde se podrá visualizar el ranking de los usuarios que pudieron resolver el juego en menor tiempo. 
 
\begin{figure}[htbp]
\begin{center}
\includegraphics[width=.40\textwidth]{./imagenes/Puntajes3.png}
\caption{Puntuaciones}
\label{Puntuaciones}
\end{center}
\end{figure} 

\ \\ Para regresar al menu anterior (Figura 4.2) presionamos en \textbf{REGRESAR}.

\begin{figure}[htbp]
\begin{center}
\includegraphics[width=.40\textwidth]{./imagenes/Regresar.png}
\caption{Salir de Score}
\label{Salir de Score}
\end{center}
\end{figure} 

\ \\Sin embargo, si no hay puntajes guardados, la aplicación mostrará el siguiente mensaje de error:

\begin{figure}[htbp]
\begin{center}
\includegraphics[width=.20\textwidth]{./imagenes/Puntajes2.png}
\caption{No hay Puntajes}
\label{No hay Puntajes}
\end{center}
\end{figure} 

\ \\Se deberá presionar en ACEPTAR para continuar.


\ \\ En el menú principal (Figura 4.2) si ahora presionamos en  \textbf{PLAY} entonces el juego iniciará y mostrará la siguiente pantalla:

\begin{figure}[htbp]
\begin{center}
\includegraphics[width=.35\textwidth]{./imagenes/Sudoku.png}
\caption{Sudoku}
\label{Sudoku}
\end{center}
\end{figure} 

\ \\ Aqui tenemos varias opciones para poder jugar una partida de Sudoku, hay 3 paneles visuales en donde se muestra el usuario registrado, el nivel de dificultad y un cronometro que registra el tiempo transcurrido desde que se inicio la partida por el usuario.

\begin{figure}[htbp]
\begin{center}
\includegraphics[width=.35\textwidth]{./imagenes/Panel.png}
\caption{Paneles de Información}
\label{Paneles de Información}
\end{center}
\end{figure} 

\ \\ Entre los controles principales del juego tenemos las opciones de: 

\begin{figure}[htbp]
\begin{center}
\includegraphics[width=.40\textwidth]{./imagenes/Controles1.png}
\caption{Controles Principales}
\label{Controles Principales}
\end{center}
\end{figure} 

\ \\ \ \\ \textbf{Iniciar la Partida:} Generá aleatoriamente el tablero del Sudoku con números ya dispuestos en algunas casillas y las demás casillas vacias, las cuales deben ser resueltas por el usuario \\ <Nota: Al inciar la partida, el CRONOMETRO empieza a contar.>  

\begin{figure}[htbp]
\begin{center}
\includegraphics[width=.50\textwidth]{./imagenes/Tablero.png}
\caption{Tablero Aleatorio}
\label{Tablero}
\end{center}
\end{figure} 

\ \\ \textbf{Borrar el Tablero:} Nos ayuda a borrar toda los números del tablero.

\begin{figure}[htbp]
\begin{center}
\includegraphics[width=.50\textwidth]{./imagenes/Tablero1.png}
\caption{Tablero Borrado}
\label{Tablero Borrado}
\end{center}
\end{figure} 

\ \\ \ \\ \ \\ \ \\
\textbf{Comprobar la Solución:} Luego de resolver el tablero, tenemos que presionar esta opción para que la aplicación nos indique si el tablero fue resuelto correctamente o no, de la siguiente manera:

\begin{figure}[htbp]
\begin{center}
\includegraphics[width=.50\textwidth]{./imagenes/SudokuNo.png}
\caption{Sudoku Incorrecto}
\label{Sudoku Incorrecto}
\end{center}
\end{figure} 

\begin{figure}[htbp]
\begin{center}
\includegraphics[width=.50\textwidth]{./imagenes/SudokuSi.png}
\caption{Sudoku Correcto}
\label{Sudoku Correcto}
\end{center}
\end{figure} 

\ \\ \ \\ \ \\ \ \\ \ \\ \ \\ \ \\
\textbf{Hacer Trampa:} Al escoger esta opción, el simulador mostrará alertas en las casillas en donde el valor digitado es correcto o incorrecto, de la siguiente manera: \\ <Verde: Correcto y Rosado: Incorrecto>


\begin{figure}[htbp]
\begin{center}
\includegraphics[width=.50\textwidth]{./imagenes/HacerTrampa.png}
\caption{Alertas}
\label{Alertas}
\end{center}
\end{figure} 

\ \\ También el simulador muestra ayudas automáticas para el usuario, en caso de que se digite un número que ya se encuentre en la fila o columna, mostrará el siguiente mensaje:

\begin{figure}[htbp]
\begin{center}
\includegraphics[width=.50\textwidth]{./imagenes/Pistas1.png}
\caption{Ayudas}
\label{Ayudas}
\end{center}
\end{figure} 

\ \\ \ \\ \ \\ \ \\ \ \\
Entre los demás controles del juego tenemos 4 opciones:

\begin{figure}[htbp]
\begin{center}
\includegraphics[width=.60\textwidth]{./imagenes/Controles2.png}
\caption{Controles Extras}
\label{Controles Extras}
\end{center}
\end{figure} 

\ \\ \textbf{1. Guardar Juego:} Guarda la partida del Sudoku, con todos los casilleros que se tienen resuelto hasta ese momento y el tiempo transcurrido. Al presionar esta opción, aparecerá el siguiente mensaje, donde se confirma que el juego se guardo correctamente. \\
<Nota: La partida se guardará con el nombe de usuario registrado al incio del juego.>
  
\begin{figure}[htbp]
\begin{center}
\includegraphics[width=.40\textwidth]{./imagenes/Guardar.png}
\caption{Juego Guardado}
\label{Juego Guardado}
\end{center}
\end{figure} 

\ \\ \ \\ \ \ \\ \ \\ 
\textbf{2. Cargar Juego:} Carga la partida del Sudoku con la misma información que tenía en el instante en el que fue guardada, es decir casilleros resueltos y tiempo transcurrido. Al presionar este botón, aparecerá la siguiente pantalla: \\ 
<Nota: Si algún usuario no ha guardado una partida con anterioridad, esta opción estará deshabilitada para dicho usuario.>

\begin{figure}[htbp]
\begin{center}
\includegraphics[width=.50\textwidth]{./imagenes/MenuCargar.png}
\caption{Cargar Juego}
\label{Cargar Juego}
\end{center}
\end{figure} 

\ \\ En la pantalla de \textbf{Cargar Juego} podemos visualizar la partida que esta guardada y debemos presionar el boton \textbf{CARGAR}, ó en caso de que el usuario ya no quisiera cargar su partida, se debe presionar el botón \textbf{REGRESAR} para volver al menú anterior.

\begin{figure}[htbp]
\begin{center}
\includegraphics[width=.50\textwidth]{./imagenes/Controles3.png}
\caption{Controles Cargar Juego}
\label{Controles Cargar Juego}
\end{center}
\end{figure} 

\ \\ \ \\ \textbf{3. Solución del Juego:} Con esta opción, el Sudoku se resolverá por completo de forma automática. \\ <Nota: Al escoger esta opción, la partida será finalizada y no se contabilizará en el ranking de puntuaciones>

\begin{figure}[htbp]
\begin{center}
\includegraphics[width=.50\textwidth]{./imagenes/JuegoResuelto.png}
\caption{Juego Resuelto}
\label{Juego Resuelto}
\end{center}
\end{figure} 

\ \\ \textbf{4. Salir del Juego:} Con este botón podemos cerrar la aplicación, o también podemos utilizar la opción de la barra de Menú para poder salir del Juego.

\begin{figure}[htbp]
\begin{center}
\includegraphics[width=.50\textwidth]{./imagenes/SeleccionMenu.png}
\caption{Menu Salir}
\label{Menu Salir}
\end{center}
\end{figure} 






\chapter{Preguntas Frecuentes}

\begin{center}

\textbf{¿Puedo subir de nivel mientras estoy jugando una partida?} \\ No, el nivel se escoge antes de iniciar la partida, por lo que se debería volver a la pantalla inicial para escoger un nuevo nivel de dificultad.
\ \\ \ \\ \ \\

\textbf{¿Al escoger la opción salir mientras estoy jugando una partida, la partida se guarda automaticamente?} \\ No, una partida solo se guarda al seleccionar la opción GUARDAR PARTIDA.
\ \\ \ \\ \ \\

\textbf{¿Si escojo la opción borrar juego, se borran solo las casillas que he resuelto?} \\ Al seleccionar esta opción se borra el contenido de todas las 81 casillas, las propuestas y las resueltas.
\ \\ \ \\ \ \\

\textbf{¿Puedo parar el tiempo en algún instante de la partida?} \\ No, el tiempo se inicia automaticamente cuando se inicia la partida y solo se detiene al finalizar la misma, o al Guardar la partida.


\end{center}


\chapter{Glosario}
\ \\
\textbf{NIVEL:} \\ El juego se subdivide en 3 niveles que diferencian la dificultad que tiene cada partida.
\ \\ \ \\ 
\textbf{PARTIDA:} \\ Flujo del juego en donde el objetivo es resolver una tabla de Sudoku llenando las casillas vacias de la tabla con numeros del 1 al 9.
\ \\ \ \\ 
\textbf{CASILLA:} \\Son las 81 celdas que conforman la tabla 9x9 del Sudoku.
\ \\ \ \\ 
\textbf{CARGAR:} \\ Abre nuevamente una partida anteriormente guardada.
\ \\ \ \\
\textbf{SCORE:} \\ Listado con las mejores puntuaciones de las partidas jugadas.


\end{document}  
